Si j'ai bien compris vos explications, la sécurité sociale rembourse un montant de base de référence qui vaut ici $193.5$\EUR{}. Si je nomme ce chiffre $b_r$, alors \textcolor{pkl}{\href{https://imgur.com/a/fOoB4QB}{d'après mon contrat}} le pourcentage de remboursement \underline{par semestre} est de $400\%$. J'appelle ce montant $p_r = 4.0$ (divisé par $100$).
\\\\\\
\textcolor{pp}{\large Remboursement de base:}\\\\
Le traitement d'après votre devis est de 5600\EUR{}, je nomme ce montant total $M_t = 5600$.\\
Sur une base de deux semestres (sans votre aide), j'ai donc droit à un remboursement de:

$$2 \cdot p_r\cdot b_r = 2 \cdot 4\cdot 193.5 = 1548\;$$\\
Cela me donne un reste à payer $\Delta = M_t - 1548 = 5600- 1548 = 4052$\EUR{}
\\\\\\
\textcolor{pp}{\large Remboursement avec votre aide:}\\\\
Sur une base de \textbf{six semestres} (avec votre aide), j'ai donc droit à un remboursement de:

$$6 \cdot p_r\cdot b_r = 6 \cdot 4\cdot 193.5 = 4644\;$$\\
Cela me donne un reste à payer $\Delta = M_t - 4644 = 5600- 4644 = 956$\EUR{}
\\\\\\
\textcolor{pp}{\large Changement de mutuelle:}\\\\
Sur la base de \textcolor{pkl}{\href{https://www.generali.fr/sites/default/files-d8/2022-04/Tableaugarantiessantesalarie_0.pdf}{la grille de référence Generali,}} j'ai l'impression que c'est la \textcolor{pkl}{\href{https://imgur.com/a/EfqjJm8}{catégorie "Prothèse dentaire amovible"}} qui s'applique, je peux donc prétendre à des taux allant de $350\%$ à $500\%$.
\\\\
Si je prends $500\%$, cela me fait $\Delta = M_t - 6 \cdot 5\cdot 193.5 = 5600 - 5805$, je suis donc remboursé intégralement.